\documentclass[paper=a4, fontsize=11pt,twoside]{article}

% -------------------------------------------------------------------- 
% General Page Layout
% --------------------------------------------------------------------
\usepackage[a4paper]{geometry} 
\usepackage[parfill]{parskip}
\setlength{\oddsidemargin}{5mm}  % Remove 'twosided' indentation
\setlength{\evensidemargin}{5mm}

% --------------------------------------------------------------------
% Encoding and Language Settings
% --------------------------------------------------------------------
\usepackage[T1]{fontenc} 
\usepackage[utf8]{inputenc}   
% encoding may need to be changed depending on the system
\usepackage[swedish]{babel} 
\usepackage{lipsum} % Lorem Ipsum

% --------------------------------------------------------------------
%  Utilities (colors, links, pictures, ect...)
% --------------------------------------------------------------------
\usepackage{xcolor}
\usepackage{hyperref}
\usepackage{graphicx}
\usepackage{amssymb}
\usepackage{epstopdf}
\usepackage[round]{natbib}
\usepackage{float}
\DeclareGraphicsRule{.tif}{png}{.png}{`convert #1 `dirname #1`/`basename #1 .tif`.png}

% -----------------------------------------------------------------------------%
% Title Page / Document Class Definitions (Please Don't Play With This)
% -----------------------------------------------------------------------------%
	
% Table of contents depth = section & subsection
\setcounter{tocdepth}{1}
																								
% Horizontal rule
\newcommand{\HRule}[1]{\rule{\linewidth}{#1}}   															
																											
% Document Number
\newcommand{\documentNumber}[1]{\centering PUSP1742#1 \\[1.0cm]}	 										
																											
% Document Version
\newcommand{\documentVersion}[1]{\centering \small{v.#1} \\[1.0cm]}

% Group Responsible
\newcommand{\documentResponsible}[1]{\centering  Ansvarig Grupp: #1}

% Document Creator Group
\newcommand{\documentCreator}[1]{\centering Uppgjord Av: #1}	 									
																										
% Title
\makeatletter \def\printtitle{ {\centering \@title\par}} \makeatother
																											
% Author .. not really used, but it can stay in case
\makeatletter \def\printauthor{ {\centering \large \@author}} \makeatother
																											
\newcommand{\grouptitlepage}[4]{ 
	\title{
		\documentNumber{#1}																						
		\documentVersion{#2}																				
		\HRule{0.5pt} \\ % Upper rule 
		\LARGE \textbf{\uppercase{#3}} \\
		\large \textbf{\uppercase{ETSF20 Grupp 2}} 
		\HRule{2pt} \\ [1.5cm]    
		\normalsize            
		\documentResponsible{#4} \\ 
		\documentCreator{#4}  
	}																							
	\maketitle																							
	\thispagestyle{empty} 																					
	\newpage 
}
% \grouptitlepage{doc number}{Version Number}{doc title}{group responsible for
% doc}
% --------------------------------------------------------------------------------%
% Title Page / Document Class Definitions (Please Don't Play With This)
% --------------------------------------------------------------------------------%


% \date{}                                            
% Activate to display a given date or keep commented for current date


% -------------------------------------------------------
% DOCUMENT START (YOU CAN IGNORE EVERYTHING ABOVE HERE)
% -------------------------------------------------------
\begin{document}

% -------------------------------------------------------
% Title Page START
% -------------------------------------------------------
\grouptitlepage
% the \# typesets a # character into the document, you will need to replace them
% in yourdocuments. This is a template, just plug in what you need between the
% {}s. Document Code Number (same as time reports)
{18}
% Document Version Number
{1.0}
% Document Title
{SSD - Systemspecifikation}
% Group Responsible For Document
{(SG) System Grupp}
% -------------------------------------------------------
% Title Page END
% -------------------------------------------------------
\tableofcontents
% WRITE THINGS BELOW HERE
\newpage
\section{Introduktion}
Detta dokument beskriver det levererade systemet  (E-KYSS) och de tillhörande dokumenten. Det beskriver även systemets begränsningar och även eventuella skillnader mellan kravspecifikationen och det levererade systemet.

\section{Levererat System}
Projektet som levereras innehåller följande delar och dokument:
\begin{center}
\begin{tabular}{|c|c|c|c|}
	\hline	
	Dokument/Del & Dokumentnummer & Version & Kommentar \\
	\hline	
	SDP & PUSP174211 & 1.0 & \\
	\hline
	SRS & PUSP174212 & 1.2 & \\
	\hline
	SVVS & PUSP174213 & 1.0 & \\
	\hline
	STLDD & PUSP174214 & 1.0 & \\
	\hline
	SVVI & PUSP174214 & 1.0 & \\
	\hline
	SDDD & PUSP174215 & 1.0 & war-fil, initierings-skript(database.sql) och java-doc\\
	\hline
	SVVR & PUSP174216 &  1.0 & \\
	\hline
	SSD & PUSP174217 & 1.0 & \\
	\hline
\end{tabular}
\end{center}
\section{Begränsningar}
\subsection{Systemets syfte}
Det huvudsakliga målet med systemet är att erbjuda ett funktionellt tidrapporteringssystem som bygger vidare på ett redan givet “BaseBlockSystem”. Detta ska kunna användas av projektgrupper för att enkelt kunna rapportera in arbetstid.
\subsection{Skillnader mellan kravspecification och levererat system}
Krav 8.0.3-6.0.6 valdes att inte implementeras.
\flushleft
Krav 8.2.13 samt 8.2.14 (projektledare skall kunna se slut och start för faser) kunde inte implementeras. I SRS:n står det även att en projektledare skapas i tre steg, vilket inte gäller för det levererade systemet. I SRS:n stod det att en användare skapas, läggs till i en projektgrupp, och sätts sedan till projektledare. I det levererade systemet kan man sätta en användare till projektledare samtidigt som man sätter in dem i en projektgrupp. Dessutom behövde så använder inte det levererade systemet HTML (vilket var specificerat i SDP:n) utan använder istället JSP, vilket genererar HTML. 
\section{Anvisningar för installation}
\begin{enumerate}
\item
Systemet måste startas på en Tomcat Server.
\item
Systemet måste vara uppkopplat till en MySQL server, vilket redan är hårdkodat i systemet.
\item
database.sql måste köras för att sätta upp MySQL-databasen.
\end{enumerate}
\end{document}
