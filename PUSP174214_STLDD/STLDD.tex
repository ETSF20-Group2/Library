\documentclass[paper=a4, fontsize=11pt,twoside]{article}

% -------------------------------------------------------------------- 
% General Page Layout
% --------------------------------------------------------------------
\usepackage[a4paper]{geometry} 
\usepackage[parfill]{parskip}
\setlength{\oddsidemargin}{5mm}  % Remove 'twosided' indentation
\setlength{\evensidemargin}{5mm}

% --------------------------------------------------------------------
% Encoding and Language Settings
% --------------------------------------------------------------------
\usepackage[T1]{fontenc} 
\usepackage[utf8]{inputenc}   
% encoding may need to be changed depending on the system
\usepackage[swedish]{babel} 
\usepackage{lipsum} % Lorem Ipsum

% --------------------------------------------------------------------
%  Utilities (colors, links, pictures, ect...)
% --------------------------------------------------------------------
\usepackage{xcolor}
\usepackage{hyperref}
\usepackage{graphicx}
\usepackage{amssymb}
\usepackage{epstopdf}
\usepackage[round]{natbib}
\usepackage{float}
\DeclareGraphicsRule{.tif}{png}{.png}{`convert #1 `dirname #1`/`basename #1 .tif`.png}

% -----------------------------------------------------------------------------%
% Title Page / Document Class Definitions (Please Don't Play With This)
% -----------------------------------------------------------------------------%
	
% Table of contents depth = section & subsection
\setcounter{tocdepth}{1}
																								
% Horizontal rule
\newcommand{\HRule}[1]{\rule{\linewidth}{#1}}   															
																											
% Document Number
\newcommand{\documentNumber}[1]{\centering PUSP1742#1 \\[1.0cm]}	 										
																											
% Document Version
\newcommand{\documentVersion}[1]{\centering \small{v.#1} \\[1.0cm]}

% Group Responsible
\newcommand{\documentResponsible}[1]{\centering  Ansvarig Grupp: #1}

% Document Creator Group
\newcommand{\documentCreator}[1]{\centering Uppgjord Av: #1}	 									
																										
% Title
\makeatletter \def\printtitle{ {\centering \@title\par}} \makeatother
																											
% Author .. not really used, but it can stay in case
\makeatletter \def\printauthor{ {\centering \large \@author}} \makeatother
																											
\newcommand{\grouptitlepage}[4]{ 
	\title{
		\documentNumber{#1}																						
		\documentVersion{#2}																				
		\HRule{0.5pt} \\ % Upper rule 
		\LARGE \textbf{\uppercase{#3}} \\
		\large \textbf{\uppercase{ETSF20 Grupp 2}} 
		\HRule{2pt} \\ [1.5cm]    
		\normalsize            
		\documentResponsible{#4} \\ 
		\documentCreator{#4}  
	}																							
	\maketitle																							
	\thispagestyle{empty} 																					
	\newpage 
}
% \grouptitlepage{doc number}{Version Number}{doc title}{group responsible for
% doc}
% --------------------------------------------------------------------------------%
% Title Page / Document Class Definitions (Please Don't Play With This)
% --------------------------------------------------------------------------------%


% \date{}                                            
% Activate to display a given date or keep commented for current date


% -------------------------------------------------------
% DOCUMENT START (YOU CAN IGNORE EVERYTHING ABOVE HERE)
% -------------------------------------------------------
\begin{document}

% -------------------------------------------------------
% Title Page START
% -------------------------------------------------------
\grouptitlepage
% the \# typesets a # character into the document, you will need to replace them
% in yourdocuments. This is a template, just plug in what you need between the
% {}s. Document Code Number (same as time reports)
{14}
% Document Version Number
{0.1}
% Document Title
{STLDD - Högnivådesign}
% Group Responsible For Document
{(UG) Utvecklingsgrupp}
% -------------------------------------------------------
% Title Page END
% -------------------------------------------------------
\tableofcontents
% WRITE THINGS BELOW HERE

\section{Introduktion}
Detta dokument beskriver högnivådesignen för ett tidrapporteringssystem, baserat på ”BaseBlockSystem”.
% TODO: Komplettera detta med lite mer introduktionell information.

\section{Referensdokument}
BaseBlockSystem STLDDD: \textbf{\textit{PUSS12004 version: 1.0}} gäller för alla punkter om det inte är specificerat i respektive underrubrik att det som står i \textbf{\textit{PUSS12004 version: 1.0}} utgår för specifik del.


\section{Översikt}
Systemet är utvecklad för applikationsservern Apache Tomcat, där Java Servlet, JavaServer Pages (JSP) samt Java Expression Language (JSP EL)-teknologier tillämpas. Systemets huvudfunktion är att fungera som tidrapporteringssystem.
% TODO: Komplettera detta med lite mer översiktig information.

\subsection{JSP}

\subsubsection{class x:} words 



%cccccc

%\begin{enumerate}
%	\item \textbf{class x} xxx
%\end{enumerate}

\paragraph{fil login.jsp}Sidan är startsida för systemet och innehåller ett inloggningsformulär för att komma åt tidrapporteringsfunktionerna.
\paragraph{fil groupmanagement.jsp}Grupphantering… % TODO: utöka
\paragraph{fil usermanagement.jsp}Användarhantering… % TODO: utöka
\paragraph{fil reportmanagement.jsp}Veckorapporthantering… % TODO: utöka
\paragraph{fil report.jsp}Veckorapportering… % TODO: utöka
\paragraph{fil dashboard.jsp}Sammanställning… % TODO: utöka
\paragraph{fil user.jsp}Användarinställningar… % TODO: utöka

\subsection{Servlet}
\paragraph{class LogIn} Används vid inloggnin… % TODO: utöka
\paragraph{class GroupManagement} Används för grupphantering… % TODO: utöka
\paragraph{class UserManagement} Används för användarhantering… % TODO: utöka
\paragraph{class ReportManagement} Används för veckorapporthantering… % TODO: utöka
\paragraph{class Report} Används för veckorapportering… % TODO: utöka
\paragraph{class Dashboard} Används för sammanställning/översikt… % TODO: utöka
\paragraph{class User} Används för användarinställningar… % TODO: utöka

\subsection{JavaBeans}
\paragraph{class GorupManagementBean} % TODO: utöka
\paragraph{class UserManagementBean} % TODO: utöka
\paragraph{class ReportManagementBean} % TODO: utöka
\paragraph{class ReportBean} % TODO: utöka
\paragraph{class DahsboardBean} % TODO: utöka
\paragraph{class UserBean} % TODO: utöka

\subsection{Övriga Java-filer} % TODO: utöka
\paragraph{class Database} % TODO: utöka
\paragraph{class DatabaseHandler} % TODO: utöka

\newpage

% Saker som bör tänkas på vid skapandet av användargränssnittet. Listan refererar till SRS PUSP174212 V.0.2 och 
% Base Block System SRS PUSS12002 V. 1.0.
%                                                              V. 2

% 
% SRS PUSP174212:
% \item[8.1.1] - 
% \item[8.1.2]
% 
% \item[9.1] Samtliga krav - 
% \item[9.2] Samtliga krav - 
% \item[9.3] Samtliga krav -


\end{document}










