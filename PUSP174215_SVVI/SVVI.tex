\documentclass[paper=a4, fontsize=11pt,twoside]{article}

% -------------------------------------------------------------------- 
% General Page Layout
% --------------------------------------------------------------------
\usepackage[a4paper]{geometry} 
\usepackage[parfill]{parskip}
\setlength{\oddsidemargin}{5mm}  % Remove 'twosided' indentation
\setlength{\evensidemargin}{5mm}

% --------------------------------------------------------------------
% Encoding and Language Settings
% --------------------------------------------------------------------
\usepackage[T1]{fontenc} 
\usepackage[utf8]{inputenc}   
% encoding may need to be changed depending on the system
\usepackage[swedish]{babel} 
\usepackage{lipsum} % Lorem Ipsum

% --------------------------------------------------------------------
%  Utilities (colors, links, pictures, ect...)
% --------------------------------------------------------------------
\usepackage{xcolor}
\usepackage{hyperref}
\usepackage{graphicx}
\usepackage{amssymb}
\usepackage{epstopdf}
\usepackage[round]{natbib}
\usepackage{float}
\DeclareGraphicsRule{.tif}{png}{.png}{`convert #1 `dirname #1`/`basename #1 .tif`.png}

% -----------------------------------------------------------------------------%
% Title Page / Document Class Definitions (Please Don't Play With This)
% -----------------------------------------------------------------------------%
	
% Table of contents depth = section & subsection
\setcounter{tocdepth}{2}
																								
% Horizontal rule
\newcommand{\HRule}[1]{\rule{\linewidth}{#1}}   															
																											
% Document Number
\newcommand{\documentNumber}[1]{\centering PUSP1742#1 \\[1.0cm]}	 										
																											
% Document Version
\newcommand{\documentVersion}[1]{\centering \small{v.#1} \\[1.0cm]}

% Group Responsible
\newcommand{\documentResponsible}[1]{\centering  Ansvarig Grupp: #1}

% Document Creator Group
\newcommand{\documentCreator}[1]{\centering Uppgjord Av: #1}	 									
																										
% Title
\makeatletter \def\printtitle{ {\centering \@title\par}} \makeatother
																											
% Author .. not really used, but it can stay in case
\makeatletter \def\printauthor{ {\centering \large \@author}} \makeatother
																											
\newcommand{\grouptitlepage}[4]{ 
	\title{
		\documentNumber{#1}																						
		\documentVersion{#2}																				
		\HRule{0.5pt} \\ % Upper rule 
		\LARGE \textbf{\uppercase{#3}} \\
		\large \textbf{\uppercase{ETSF20 Grupp 2}} 
		\HRule{2pt} \\ [1.5cm]    
		\normalsize            
		\documentResponsible{#4} \\ 
		\documentCreator{#4}  
	}																							
	\maketitle																							
	\thispagestyle{empty} 																					
	\newpage 
}
% \grouptitlepage{doc number}{Version Number}{doc title}{group responsible for
% doc}
% --------------------------------------------------------------------------------%
% Title Page / Document Class Definitions (Please Don't Play With This)
% --------------------------------------------------------------------------------%


% \date{}                                            
% Activate to display a given date or keep commented for current date


% -------------------------------------------------------
% DOCUMENT START (YOU CAN IGNORE EVERYTHING ABOVE HERE)
% -------------------------------------------------------
\begin{document}

% -------------------------------------------------------
% Title Page START
% -------------------------------------------------------
\grouptitlepage
% the \# typesets a # character into the document, you will need to replace them
% in yourdocuments. This is a template, just plug in what you need between the
% {}s. Document Code Number (same as time reports)
{15}
% Document Version Number
{0.0}
% Document Title
{SVVI - Software Verification and Validation Intstructions}
% Group Responsible For Document
{(TG) Test Grupp}
% -------------------------------------------------------
% Title Page END
% -------------------------------------------------------
\tableofcontents
% WRITE THINGS BELOW HERE
\section{Introduktion}

I detta dokument kommer detaljerade testinstruktioner presenteras.

\section{Referensdokument}

\begin{enumerate}
\item Software Verification and validation Specification PUSP174213 V 0.2
\item Software Verification and validation Specification PUSS12003 V 1.0
\end{enumerate}

\section{Testinstruktioner}

De detaljerade testinstruktionerna för testfallen i ref. 1 visas nedan.
Testfallen är dokumenterade steg för steg som skall genomföras av testaren. 
Flertalet av instruktionerna har förkrav som måste vara uppfyllda för att testet
skall kunna genomföras korrekt samt slutkrav som ska vara uppfyllda för att
testet ska kunna ses som godkänt. Alla funktionstester utgår från systemets
startsida om inget ”Anna”t anges.



 \subsection {Ändring}
En ändring är då:

\begin{enumerate}
  \item En medlem läggs till/tas bort i gruppen
  \item      En medlem i gruppen tilldelas en ny roll
  \item      PG signerar tidrapport
\end{enumerate}


\subsection{Administratör}

\subsubsection*{FT1 Kan skapa projektgrupper}
Förkrav: Admin är inloggad.
\newline
Slutkrav:  En ny projektgrupp (projektgrupp 2) har skapats.

\begin{enumerate}
	\item Admin går till sidan för hantering av projektgrupper.
	\item Admin väljer att lägga till projektgrupp.
	\item Admin uppmanas att mata in namn på projektgrupp som ska skapas.
	\item Admin matar in namnet ``"projektgrupp 2".
	\item En projektgrupp med namnet ``"projektgrupp 2" har skapats.
	\item ``"Projektgrupp 2" har lagts till på sidan för hantering av
	projektgrupper.
	\item Admin ser en uppdatering av sidan för hantering av projektgrupper.
\end{enumerate}

\subsubsection*{FT2 Kan ta bort projektgrupper}
Förkrav: Förkrav: Admin är inloggad, det finns minst en projektgrupp i systemet
däribland “projektgrupp 2”, “projektgrup 3” och “projektgrupp 4” existerar.
\newline
Slutkrav: “Projektgrupp 2” och "projektgrupp 4" har tagits bort från
systemet.
\begin{enumerate}
	\item Admin går till sidan för hantering av projektgrupper.
	\item Admin väljer vilka/vilken projektgrupp(projektgrupp 2 och projektgrupp 4)
	som ska tas bort genom att markera med kryss i rutor.
	\item Admin trycker på en knapp för att ta bort valda
	projektgrupper(projektgrupp 2 och projektgrupp 4).
	\item En varningsruta där admin ombeds bekräfta borttagningen av projektgrupper
	visas för admin. Admin trycker ok på varningsrutan.
	\item De grupper som har raderats finns inte kvar i systemet.
	\item Admin ser en uppdaterad version av sidan för hantering av projektgrupper.
	\item Den/de grupper som har tagits bort (projektgrupp 2 och projektgrupp 4)
	finns ej med på den uppdaterade sidan.
\end{enumerate}

\subsubsection*{FT3 Tilldelar rollen PG i en projektgrupp}
	Förkrav: Admin är inloggad, det finns en projektgrupp med namnet “projektgrupp 2” ,det finns ingen användare i “projektgrupp 2”.
	\newline
	Slutkrav:  Användaren som skapats har tilldelats rollen som PG för gruppen
	“projektgrupp 2”.
	
	\begin{enumerate}
		\item Admin går till sidan för att hantera användare.
		\item Admin lägger till en användare ”Anna” till systemet och till “projektgrupp 2”.
		\item Admin tilldelar rollen PG i “projektgrupp 2” till ”Anna”.
		\item ”Anna” får rollen som PG för “projektgrupp 2”.
		\item Admin ser en uppdaterad sida för medlemmar i projektet med ”Anna” som PG
	\end{enumerate}
	
	Förkrav: Admin är inloggad, det finns minst en projektgrupp med namn
	“projektgrupp 2”, det finns en användare ”Anna” i “projektgrupp 2”.
	\newline
	Slutkrav: ”Anna” har tilldelats rollen PG i “projektgrupp 2”.
 	
 	\begin{enumerate}
	\item Admin går till sidan för hantering av användare för “projektgrupp 2”.
	\item Admin väljer ”Anna” bland användarna i projektgrupp  och tilldelar rollen
	PG till henne.
	\item ”Anna” får rollen PG i “projektgrupp 2”.
	\item Admin ser en uppdaterad sida för hantering av användare i “projektgrupp 2” med ”Anna” som PG.
		
	\end{enumerate}
	
	\subsubsection*{FT4 Lägger till och tar bort användare från systemet}
	
	Förkrav: Admin är inloggad, det finns ingen användare i systemet,användaren som
	ska läggas till har en giltig epost som admin har kännedom om.
	\newline
	Slutkrav: Användare ”Anna” finns i systemet.
	
	\begin{enumerate}
	  \item Admin går till sidan för hantering av användare.
	  \item Admin väljer att lägga till en användare.
	  \item Admin väljer att mata in “”Anna”” som användarens namn.
	  \item Admin väljer att mata in  ”Anna”.Puss@Spray.se som epost address.
		\item Admin ser en uppdatering av sidan över användare där den nya användaren
		”Anna” är tillagd.
	\end{enumerate}
	
	Förkrav: Admin är inloggad, det finns en “projektgrupp 2” i systemet, användare 
	”Anna” finns i projektgruppen.
	\newline
	Slutkrav: Användaren ”Anna” finns inte i systemet längre.
	
	\begin{enumerate}
	 \item Admin går till sidan för hantering av användare.
   \item Admin markerar användaren ”Anna”.
   \item Admin väljer att radera användaren.
   \item Admin ser en uppdatering av sidan utan användaren ”Anna”.
	  
	\end{enumerate}
	
	Förkrav: Admin är inloggad, det finns en “projektgrupp 2” i systemet, användarna 
	”Anna”,Nicklas och ”Sven" finns i projektgruppen.
	\newline
	Slutkrav: Användarna ”Anna”,Nicklas,”Sven" finns inte i systemet längre.
	
	\begin{enumerate}
	\item  Admin går till sidan för hantering av användare.
    \item Admin markerar användaren ”Anna”, Nicklas och ”Sven".
    \item Admin väljer att radera användare.
    \item Admin ser en uppdatering av sidan utan användaren ”Anna”, Nicklas och
  ”Sven".
	  
	\end{enumerate}
	
	
	
	\subsubsection*{FT5 Kan ej ta bort sig själv }
	
	Förkrav: Admin är inloggad.
	\newline
	Slutkrav: Admin finns kvar i systemet.
	
	\begin{enumerate}
	\item Admin går till sidan för hantering av användare
 	\item Admin hittar inte Admin i listan av användare
 	\end{enumerate}
	
	
	\subsection{Projektledare}
	
	\subsubsection*{FT6 Kan endast tilldela följande roller till projektmedlemmar i
	projektgruppen }
	
	Förkrav: Projektledare är inloggad, det finns en “projektgrupp 2” i systemet,
	användare  ”Anna” finns i systemet och tillhör “projektgrupp 2”.
	\newline
	Slutkrav: Projektmedlem ”Anna” blir tilldelad  rollen TG.
	
	\begin{enumerate}
	 \item Projektledare klickar på “Main”.
	\item En ny sida visas där projektledaren kan se “Organize group”.
	\item Projektledare klickar på “Organize group”.
	\item En ny sida visas med en lista på medlemmar i projektgruppen.
	\item Projektledare klickar på roll-alternativet på medlemmen ”Anna”.
	\item Projektledare tilldelar rollen TG.
	\item Projektledare trycker på knappen “Update group”.
	\item Projektledare ser en uppdatering av sidan som visas i steg 4. 
	  
	\end{enumerate}

	\subsubsection*{FT7 Kan ändra roller i projektgruppen }
	
	Förkrav: Projektledare är inloggad, det finns en “projektgrupp 2” i systemet,
	användare  ”Anna” finns i systemet och tillhör “projektgrupp 2”. ”Anna” har rollen TG.
	\newline
	Slutkrav: Projektmedlem ”Anna” får sin roll ändrad.
	
	\begin{enumerate}
	\item Projektledare klickar på “Main”.
	\item Projektledare klickar på “Organize group”.
	\item En ny sida visas med en lista på medlemmar i projektgruppen.
	\item Projektledare klickar på roll-alternativet på medlemmen ”Anna”.
	\item Projektledare ändrar ”Annas” roll från TG till SG.
	\item Projektledare trycker på knappen “Update group”.
	\item Projektledare ser en uppdatering av sidan som visas i steg 3 nu med ”Anna”
	i en ny roll.
	\item Punkt 1-7 upprepas för samtliga roller (TG,SG,UG,PG).
	  
	\end{enumerate}
	
	\subsubsection*{FT8 Kommer åt sin meny}
	Förkrav: Projektledare är inloggad.
	\newline
	Slutkrav: Projektledaren ser sin meny och kan interagera med den.
	
	\begin{enumerate}
	  \item Projektledare klickar på “Main”.
	  \item Projektledare ser en ny sidan som visar en meny.
	  \item Projektledaren provar att klicka på “Organise group”.
      \item Projektledaren ser en nya sida där innehållet från “Organise group”
      visas. 
      \item Punkterna 1-4 upprepas för samtliga alternativ i menyn som är
      Organize Group ,Sign report ,Unsign reports och View all reports.
	\end{enumerate}
	
	\subsubsection*{FT9 Försöker tilldela fler roller till samma projektmedlem}
	
	Förkrav: Projektledare är inloggad, det finns en “projektgrupp 2” i systemet,
	användare  ”Anna” och tillhör “projektgrupp 2”. ”Anna” har rollen SG.
	\newline
	Slutkrav: Projektmedlem ”Anna” har kvar sin roll som SG i systemet.
	
	\begin{enumerate}
	 \item Projektledare klickar på “Main”.
	\item Projektledare klickar på “Organize group”.
	\item En ny sida visas med en lista på medlemmar i projektgruppen.
	\item Projektledare klickar på roll-alternativet på medlemmen ”Anna”.
	\item Projektledare ändrar ”Annas” roll från SG till “TG,PG“.
	\item Projektledare trycker på knappen “Update group”.

	  
	\end{enumerate}

	\subsubsection*{FT10 Kan attestera rapport}
		Förkrav: Projektledare är inloggad, det finns en tidrapport i systemet från
		användaren  ”Anna” som tillhör “projektgrupp 2”.
		\newline
		Slutkrav: ”Annas” tisrapport blir attesterad.
		
		\begin{enumerate}
		  \item Projektledare klickar på “Main” / “ Time Reporting”.
		  \item Projektledare klickar på “Sign Time Reports”.
		  \item En lista kommer upp med alla Veckorapporter.
		  \item Projektledare  markerar ”Annas” rapport och klickar på knappen “Sign”.
		  \item När projektledaren signerat rapporten, uppdateras sidan och visar en
	uppdaterad lista av alla veckorapporter.
		  
		\end{enumerate}
		
		Förkrav: Projektledare är inloggad, det finns en tidrapport i “systemet” från
		användaren  ”Anna” och ”Sven” som tillhör “projektgrupp 2”.
		\newline
		Slutkrav: ”Annas” och ”Svens” tisrapport blir attesterad.
		
		\begin{enumerate}
    		\item Projektledare klickar på “Main”.
			\item Projektledare klickar på “Sign Reports”.
			\item En lista kommer upp med alla veckorapporter.
			\item Projektledare  markerar ”Annas” och ”Sven"s rapporter och klickar på
			knappen “Sign”.
			\item När projektledaren signerat rapporten, uppdateras sidan och visar en
			uppdaterad lista av alla veckorapporter.

				  

		\end{enumerate}
		
		\subsubsection*{FT11 Kan annulera en attesterad rapport}
		
		Förkrav: Projektledare är inloggad, det finns minst en attesterad tidrapport i
		systemet från användaren  ”Anna” som tillhör “projektgrupp 2”.
		\newline
		Slutkrav: ”Annas” tidrapport blir annulerad.
		
		\begin{enumerate}
		  \item Projektledare klickar på “Main”.
		  \item Projektledare klickar på “Unsign Reports”.
		  \item En lista kommer upp med alla attesterade rapporter.
		  \item Projektledare markerar ”Annas” rapport och klickar på knappen “Submit
		  changes” .
		  \item Vald rapport från ”Anna” har annulerats.
		  
		\end{enumerate}
		
		Förkrav: Projektledare är inloggad, det finns minst en attesterad tidrapport i
		“systemet” från användaren  ”Anna” och ”Sven” som tillhör “projektgrupp 2”.
		\newline
		Slutkrav: ”Annas” och ”Sven"s tidrapport blir annulerad.
		
		\begin{enumerate}
		 \item Projektledare klickar på “Main”.
		 \item Projektledare klickar på “Unsign Reports”.
		 \item En lista kommer upp med alla attesterade rapporter.
		 \item Projektledare markerar ”Annas” och ”Svens" rapport och klickar sedan på
		 knappen “Submit changes”.
		  \item Valda rapporter har nu annulerats.
		  
		\end{enumerate}
		
		\subsubsection*{FT12 Kan se allas statistik i sin projektgrupp}
		
		Förkrav: Det finns minst två projektmedlemmar i gruppen (projektgrup 2).
		Inklusive projektledaren.
		\newline
		Slutkrav: Projektledaren ser statistiken sorterad enligt de val som gjorts i listan.
		\begin{enumerate}
		\item Projektledaren går till statistiksidan
		\item Projektledaren väljer de översta valet i ref 1 appendix B 9.3
		\item Projektledaren ser statistiken i sin grupp sorterad enligt nyss gjorda
		valet 
		\item Steg 1-3 upprepas för alla punkterna i ref 1 appendix B 9.3
		
		\end{enumerate}		
		
		\subsection{Användare}
		
		\subsubsection*{FT13 Kan lämna in tidrapport}
		
		Förkrav: Användaren ”Anna” är inloggad.
		\newline 
		Slutkrav: Användare har lämnat in en tidrapport.
		
		\begin{enumerate}
		  \item ”Anna” går till sida för hantering av tidrapport.
			\item ”Anna” ser en sammanställning med alla tidigare rapporterad tid.
			\item ”Anna” trycker på knappen “ny tidrapport“.
			\item ”Anna” fyller i tidrapporten enligt den tid hon har arbetat under den
			senaste veckan.
			\item Användaren trycker på knappen föra att skicka
			tidrapport.
			\item Användaren ser en uppdatering av sidan.
		  
		\end{enumerate}
		
		\subsubsection*{FT14 Kan uppdatera sin tidrapport innan attestering}
		
		Förkrav: Användaren ”Anna” är inloggad. ”Anna” har en osignerade rapport inlämnad. 
		\newline
		Slutkrav: ”Anna” har ändrat en tidrapport.
		
		\begin{enumerate}
		 \item ”Anna” går till sida för hantering av tidrapport(“Time reporting”?).
			\item ”Anna” trycker på knappen ska “Edit Time Report“.
			\item ”Anna” ser en sida med alla osignerade tidrapporter.
			\item ”Anna” klickar på den senaste tidrapporter.
			\item ”Anna” ser tidrapporten med det ifyllda värdena.
			\item ”Anna” ändrar 60 till 120.
			\item ”Anna” trycker på “Submit Changes”.
			\item Ändringarna sparas.
			\item En uppdaterad sidan visas över alla osignerade tidrapporter.
		  
		\end{enumerate}
		
		\subsubsection*{FT15 Kan se sin statistik}
		Förkrav: Användaren ”Anna” är inloggad, ”Anna” har minst en ifylld tidrapport.
		\newline
		Slutkrav: Användaren ser sin statistik.
		
		\begin{enumerate}
		  \item Användaren klickar på ”Time Reporting” i menyn. 
		  \item En ny sida visas som innehåller alternativ för tidrapportering samt en
		  sammanställning för den totalt rapporterade tiden för ”Anna” i form av en
		  tabell.
		  
		\end{enumerate}
		
		\subsubsection*{FT16 Ändra lösenord}
		
		Förkrav: Användaren ”Anna” är inloggad med lösenord password1234.
		\newline 
		Slutkrav: Användare har ändrat lösenordet.
		
		\begin{enumerate}
		  	\item Användaren trycker på ``"Change Password''' i menyn.
			\item En ny sida visas med tre textboxar.
			\item Användaren fyller i nuvarande lösenord(skriv ut lösenord) i första
			textrutan.
			\item Användaren fyller i password1234 i textruta 1.
			\item Användaren fyller i password1234 i textruta 2. 
			\item Användaren trycker på “Change password”.
			\item Användaren ser en ny sida där det står att lösenord är bytt.
		  
		\end{enumerate}
		
		
		
		Förkrav: Användare är inloggad med lösenord password1234.
		\newline 
		Slutkrav: Användare har ändrat lösenordet.
		
		\begin{enumerate}
		\item Användare trycker på Change Password i menyn.
		\item	En ny sida visas med tre textboxar.
		\item	Användaren fyller i nuvarande lösenord(skriv ut lösenord) i första
		textrutan.
		\item Användaren fyller i fel lösenord (adminp) i textruta 1.
		\item	Användaren upprepar i textruta 2.
		\item	Användaren trycker på “Change password”.
		\item	Användaren ser ett felmeddelande.
		\item	Användare skickar till steg 2.
		  
		\end{enumerate}
		
		\subsection{Projektkrav}
		
		\subsubsection*{FT17 Systemet uppfyller användarbarhetskraven}
		
		Ref 2 sektion 5
		
		
		
		\subsubsection*{FT18 Systemet är en vidarekoppling av grundsystemet}
	
		Ref 2
	
		\subsubsection*{ FT19 Loggfilen raderas när en projektgrupp raderas från
		systemet}

		Förkrav:  Projektgruppen “Loggtest ” finns i systemet, admin är inloggad.
		\newline
		Slutkrav: Projektgruppen “Loggtest” finns inte längre I systemet och loggfilen
		som var kopplad till gruppen finns inte heller kvar i systemet.
		
		
		\begin{enumerate}
		  \item       Admin tar bort projektgruppen “Loggtest”
		  \item      Admin väljer “Loggfile” i menyn
		  \item       En sida där admin uppmanas att välja vilken loggfil som ska visas genom att kryssa i en ruta
		  \item      Loggfilen för grupen “Loggtest” finns inte i listan
		  
		  
		\end{enumerate}
		
		
		
		\subsubsection*{ FT20 Ett inlägg I loggfilen innehåller tidsstämpel samt
		användare}
		
		Förkrav: Projektgruppen “Loggtest ” finns i systemet, minst en ändring har
		gjorts i gruppen sedan den skapades, admin är inloggad.
		\newline
		Slutkrav: Loggfilen visas för addmin, varje inlägg I loggfilen innehåller en
		tidsstämpel samt namnet på en användare.
		
		
		\begin{enumerate}
		  \item     Admin väljer “Loggfile” i menyn
		  \item      En sida där admin uppmanas att välja vilken loggfil som ska visas genom att kryssa i en ruta
		  \item    Admin väljer att visa loggfilen för den nyligen skapade gruppen “Loggtest”
		  \item     Den valda loggfilen läses in och visas för admin
		  \item      Admin kan se när och av vem som varje ändringar i gruppen har
		  gjorts.
		  
		\end{enumerate}
		
		\subsubsection*{ FT21  Loggfilen för projektgruppen uppdateras för varje
		ändring I projektgruppen}

		Förkrav:  Projektgruppen “Loggtest ” finns i systemet, minst en ändring har
		gjorts i gruppen sedan den skapades, admin är inloggad.
		\newline
		Slutkrav: Loggfilen visas för addmin med ett inlägg för varje ändring som har skett samt ett från då gruppen skapades.
		
		\begin{enumerate}
		  \item     Admin väljer “Loggfile” i menyn
		  \item         En sida där admin uppmanas att välja vilken loggfil som ska visas genom att kryssa i en ruta
		  \item         Admin väljer att visa loggfilen för den nyligen skapade gruppen “Loggtest”
		  \item         Den valda loggfilen läses in och visas för admin
		  \item          Loggfilen visar alla ändringar som har gjorts i gruppen.
		  
		\end{enumerate}
		
		
		
		\subsubsection*{ FT22  Ny loggfil skapas när ny projektgrupp skapas}
		
		Förkrav: Det finns inte någon projektgrupp med namnet “Loggtest”, admin är
		inloggad.
		\newline
		Slutkrav:  Loggfilen visas för admin med ett inlägg från tidpunkten
		när projektgruppen skapades av användare admin.
		
		\begin{enumerate}
		  \item       Admin skapar en ny projektgrupp med namnet “Loggtest”
		  \item       Admin väljer “Loggfile” i menyn
		  \item       En sida där admin uppmanas att välja vilken loggfil som ska visas genom att kryssa i en ruta
		  \item       Admin väljer att visa loggfilen för den nyligen skapade gruppen “Loggtest”
		  \item       Den valda loggfilen läses in och visas för admin
		  \item       Det enda inlägget i loggfilen är från när
		  gruppen (Loggtest) skapades av admin.
		  
		\end{enumerate}
		
		

\end{document}


